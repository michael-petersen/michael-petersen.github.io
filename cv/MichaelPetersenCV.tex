%%%%%%%%%%%%%%%%%%%%%%%%%%%%%%%%%%%%%%%%%
% Medium Length Graduate Curriculum Vitae
% LaTeX Template
% Version 1.1 (9/12/12)
%
% This template has been downloaded from:
% http://www.LaTeXTemplates.com
%
% Original author:
% Rensselaer Polytechnic Institute (http://www.rpi.edu/dept/arc/training/latex/resumes/)
%
% Important note:
% This template requires the res.cls file to be in the same directory as the
% .tex file. The res.cls file provides the resume style used for structuring the
% document.
%
%%%%%%%%%%%%%%%%%%%%%%%%%%%%%%%%%%%%%%%%%

%----------------------------------------------------------------------------------------
%	PACKAGES AND OTHER DOCUMENT CONFIGURATIONS
%----------------------------------------------------------------------------------------

\documentclass[margin, 12pt]{res} % Use the res.cls style, the font size can be changed to 11pt or 12pt here

\usepackage{times}

%\usepackage{helvet} % Default font is the helvetica postscript font
%\usepackage{newcent} % To change the default font to the new century schoolbook postscript font uncomment this line and comment the one above

\usepackage{hyperref}

\usepackage{url}

\setlength{\resumewidth}{4.7in} % Text width of the document

\begin{document}

\setlength{\parskip}{4pt}


\setlength{\sectionwidth}{1.3in}


%----------------------------------------------------------------------------------------
%	NAME AND ADDRESS SECTION
%----------------------------------------------------------------------------------------

\moveleft.5\hoffset\centerline{\huge\bf Michael S. Petersen} % Your name at the top
 
\moveleft\hoffset\vbox{\hrule width 6.5in height 1pt}\smallskip % Horizontal line after name; adjust line thickness by changing the '1pt'
 
\moveleft.5\hoffset\centerline{University of Massachusetts at Amherst Department of Astronomy} % Your address
\moveleft.5\hoffset\centerline{710 N. Pleasant St. Amherst, MA 01003}
\moveleft.5\hoffset\centerline{\href{mailto:mpete0@astro.umass.edu}{mpete0@astro.umass.edu}}


%----------------------------------------------------------------------------------------

\begin{resume}

%----------------------------------------------------------------------------------------
%	OBJECTIVE SECTION
%----------------------------------------------------------------------------------------
 
%\section{OBJECTIVE}  

%A position in the field of computers with special interests in business applications programming, information processing, and management systems. 

%----------------------------------------------------------------------------------------
%	EDUCATION 
%----------------------------------------------------------------------------------------

\section{\sc Education \& Awards}

{\bf Doctor of Philosophy,} Astronomy \hfill 2018\\
\hspace*{0.25 in} {\sl The non-linear dynamics of barred galaxy evolution in $\Lambda$CDM}\\
\hspace*{0.25 in} {\sl Advisors: Martin D. Weinberg, Neal Katz}\\
{\bf Master of Science,} Astronomy \hfill 2012\\
\hspace*{0.25 in} {\sl UMass Amherst, Amherst, MA}\\
\hspace*{0.4 in} Mary Dailey Irvine Travel Grant \hfill 2016, 2017, 2018\\
\hspace*{0.4 in} Massachusetts Space Grant \hfill 2010, 2011, 2015\\
{\bf Bachelor of Arts,} Astronomy \& Physics, Music \hfill 2010\\
\hspace*{0.25 in} {\sl Colgate University, Hamilton, NY}\\
\hspace*{0.4 in} Astronomy \& Physics Honors

 
%----------------------------------------------------------------------------------------
%	RESEARCH
%----------------------------------------------------------------------------------------

\section{\sc Research} 

%{\bf Graduate Research Assistant}\\
%\hspace*{0.25 in} {\sl Martin D. Weinberg and Neal Katz, University of Massachusetts} \\
Design, implement, execute, and analyze precision numerical models to understand dynamical evolution in disk galaxies.
 
%----------------------------------------------------------------------------------------
%	PUBLICATIONS
%----------------------------------------------------------------------------------------

\section{\sc Publications} 
 \underline{\sl Refereed Publications} \\
{\bf Petersen, M.S.}, Katz, N. , \& Weinberg, M.D. \href{http://adsabs.harvard.edu/abs/2016PhRvD..94l3013P}{{\it The Dynamical Response of Dark Matter to Galaxy Evolution Affects Direct-Detection Experiments}}, Phys Rev D, 2016\\
{\bf Petersen, M. S.}, Weinberg, M. D., and Katz, N. \href{http://adsabs.harvard.edu/abs/2016MNRAS.463.1952P}{{\it  Dark matter trapping by stellar bars: the shadow bar}} MNRAS, 463:1952–1967.\\
Bary, Jeffrey S. \& {\bf Petersen, M. S.} \href{http://adsabs.harvard.edu/abs/2014ApJ...792...64B}{{\it Anomalous Accretion Activity and the Spotted Nature of the DQ Tau Binary System}} MNRAS, 463:1952–1967.

\vspace*{8pt}
%\underline{\sl Conference Proceedings} \\
%{\bf Petersen, M. S.}, Weinberg, M. D., and Katz, N. \href{http://adsabs.harvard.edu/abs/2014ASPC..480..157P}{{\it 
%Creation of Peanut-Shaped Bulges via the Slow Mode of Bar Growth}} ASPC, 480:157.

\underline{\sl Publications In Preparation} \\
{\bf Petersen, M. S.}, Weinberg, M. D., and Katz, N. {\it Using commensurabilities and orbit structure to understand barred galaxy evolution}\\
{\bf Petersen, M. S.}, Weinberg, M. D., and Katz, N. {\it Using angular momentum and torque to understand barred galaxy evolution} \\
{\bf Petersen, M. S.}, Weinberg, M. D., and Katz, N. {\it Using harmonic decomposition to understand barred galaxy evolution} \\
{\bf Petersen, M. S.}, Weinberg, M. D., and Katz, N. {\it Understanding trends in bar formation and evolution in varied $n$-body models} \\
{\bf Petersen, M. S.}, Weinberg, M. D., and Katz, N. {\it Using the bar to drive radial mixing and bulge formation in a Milky Way-like galaxy} \\

%{\bf Petersen, M. S.}, Gutermuth, R.A., and Wilson, G.W. {\it Circumstellar Disk Masses in IC~348: Newly Detected Disks from LMT/AzTEC} \\
\pagebreak

\section{\sc Selected Meetings \& Conferences}
{\bf AAS 231, January 2018}, Washington, D.C.\\
Dissertation: {\it A Modern Picture of Barred Galaxy Dynamics}

{\bf APS April Meeting, January 2017}, Washington, D.C.\\
Contributed Talk: {\it The Dynamical Response of Dark Matter to Galaxy Evolution Affects Direct-Detection Experiments}

%{\bf Great Lakes Cosmology Workshop, June 2016}, Hamilton, ON\\
%Contributed Talk: {\it Dark Matter Trapping by Stellar Bars: The Shadow Bar}

{\bf What Shapes Galaxies?, April 2016}, STScI, Baltimore, MD\\
Contributed Poster: {\it Dark Matter Trapping by Stellar Bars: The Shadow Bar}

%{\bf Structure and Dynamics of Disk Galaxies, August 2013}, Petit Jean Mountain, AR\\
%Contributed Talk: {\it Creation of Peanut-Shaped Bulges via the Slow Mode of Bar Growth}

\section{\sc Selected Observational Experience}

{\bf Large Millimeter Telescope}, PI (Early Science 2,3,4 2014-2016), 60 hours\\
{\it Circumstellar Disk Masses in IC~348}\\
Using AzTEC, a 1.1 mm camera.

{\bf KPNO 0.9m}, PI (2016-2017), 5 nights; Co-I (2014-2018), 30 nights\\
{\it Deep Imaging of Nearby Low Surface Brightness Disks}\\
{\it Ionization States of Green Pea Galaxies}\\
Using the Half-Degree Imager (HDI).
% would be nice to put the Kim title in here as well

%{\bf Gemini South}, Co-I (2017A), 18 hours\\
%{\it High-Resolution Spectroscopy of Orbitally-Modulated Accretion Activity in Pre-Main Sequence Binaries.} Using PHOENIX.

%{\bf APO 3.5 m}, Co-I (2017B), 4 nights\\
%{\it Optical to near-infrared Monitoring of Eccentric Pre-Main
%  Sequence Binaries.} Using ECHELLE and TripleSpec.

%{\bf NASA IRTF}, Co-I (2018B), 5 nights\\
% {title?}


\section{\sc Teaching, Mentoring, \& Outreach}

{\bf Five College Astronomy Teaching Assistant, 2014-2018}\\
Assisted Professor James Lowenthal, Dr. Anne Jaskot, and Dr. Kim Ward-Duong in Observational Techniques I/II, a two-semester observing course involving a yearly observational component at Kitt Peak National Observatory.

{\bf Continuing Education Head Instructor, 2013-2018}\\
Developed a two-week intensive continuing education class for the University of Massachusetts, then reshaped the class into a program involving many graduate students.

{\bf Research Mentor, 2016-2017}\\
Mentored and advised Amelia Vinciguerra (UMass 2017) in her senior undergraduate research project in collaboration with Martin Weinberg.

%{\bf Public Outreach Lecturer, 2013-present}\\
%Advised the undergraduate Astronomy club. Lead semesterly star walks
%for dormitories. Offered yearly open night sky observing sessions.

{\bf Python Instructor, 2014-present}\\
Provided Python instruction and support to the undergraduate Five College summer undergraduate researchers.

\section{\sc Professional Links}
{\bf Research Webpage} \url{https://people.umass.edu/mpete0/}

{\bf Github Code Repository} \url{https://github.com/michael-petersen}


\section{\sc References}
{\bf Martin D. Weinberg}\\
Primary dissertation advisor.

{\bf Neal Katz}\\
Co-dissertation advisor.

%{\bf Mauro Giavalisco}\\
%Dissertation committee member.

%\section{\sc Professional Affiliations }

\end{resume}
\end{document}