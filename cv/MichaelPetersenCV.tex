%%%%%%%%%%%%%%%%%%%%%%%%%%%%%%%%%%%%%%%%%
% Medium Length Graduate Curriculum Vitae
% LaTeX Template
% Version 1.1 (9/12/12)
%
% This template has been downloaded from:
% http://www.LaTeXTemplates.com
%
% Original author:
% Rensselaer Polytechnic Institute (http://www.rpi.edu/dept/arc/training/latex/resumes/)
%
% Important note:
% This template requires the res.cls file to be in the same directory as the
% .tex file. The res.cls file provides the resume style used for structuring the
% document.
%
% COMPILE WITH xelatex MichaelPetersenCV.tex
%
%%%%%%%%%%%%%%%%%%%%%%%%%%%%%%%%%%%%%%%%%


\documentclass[margin, 11pt]{res} % Use the res.cls style, the font size can be changed to 11pt or 12pt here

\usepackage{times} % fallback font

\usepackage{hyperref}

\usepackage{url}

\setlength{\resumewidth}{4.7in} % Text width of the document

% https://www.overleaf.com/learn/latex/Using_colours_in_LaTeX
\usepackage[dvipsnames]{xcolor}
% now define a few colors to join webpage themes
\definecolor{blueshade}{RGB}{15, 29, 60}
\definecolor{redshade}{RGB}{172,54,56}
\definecolor{grey}{gray}{0.6}

% https://www.overleaf.com/learn/latex/Questions/I_have_a_custom_font_I%27d_like_to_load_to_my_document._How_can_I_do_this%3F
%steps to get the Roboto font: download the package from Google Fonts
% move the Regular, Italic, Bold fonts to the CV folder:
\usepackage{fontspec}
\setmainfont{Roboto}[Extension = .ttf, UprightFont = *-Regular, ItalicFont = *-Italic, BoldFont = *-Bold]

\begin{document}


\setlength{\parskip}{4pt}


\setlength{\sectionwidth}{1.2in}

%----------------------------------------------------------------------------------------
%	header
%----------------------------------------------------------------------------------------

\moveleft.5\hoffset\centerline{\huge\bf \textcolor{blueshade}{Michael S. Petersen} }
 
\moveleft\hoffset\vbox{\color{grey}\hrule width 6.5in height 1pt}\smallskip

%{ \color{grey}\rule{\linewidth}{0.5mm} }
 

\moveleft.5\hoffset\centerline{Institute for Astronomy, University of Edinburgh, Royal Observatory}
\moveleft.5\hoffset\centerline{Blackford Hill, Edinburgh EH9 3HJ, UK}
\moveleft.5\hoffset\centerline{\textcolor{grey}{\href{mailto:michael.petersen@roe.ac.uk}{michael.petersen@roe.ac.uk}}}

\begin{resume}

\section{\sc \textcolor{redshade}{Research} }

%{\bf Graduate Research Assistant}\\
%\hspace*{0.25 in} {\sl Martin D. Weinberg and Neal Katz, University of Massachusetts} \\
Design, implement, execute, and analyse precision numerical models to understand dynamical evolution in disk galaxies and their halo environs.


\section{\sc \textcolor{redshade}{Position}}

{\bf Postdoctoral Research Associate,} working with Jorge Pe{\~n}arrubia. \hfill \textcolor{grey}{2019-}
% add name of consolidated grant case

\section{\sc \textcolor{redshade}{Education}}

{\bf Doctor of Philosophy,} Astronomy \hfill \textcolor{grey}{2019}\\
\hspace*{0.25 in} {\sl \textcolor{blueshade}{The non-linear dynamics of barred galaxy evolution in $\Lambda$CDM}}\\
\hspace*{0.25 in} {\sl Advisors: Martin D. Weinberg, Neal Katz}\\
\hspace*{0.4 in} Mary Dailey Irvine Grant \hfill \textcolor{grey}{2016, 2017, 2018}\\
%\hspace*{0.4 in} Massachusetts Space Grant \hfill 2010, 2011, 2015\\
{\bf Bachelor of Arts,} Astronomy \& Physics, Music \hfill \textcolor{grey}{2010}\\
\hspace*{0.25 in} {\sl Colgate University, Hamilton, NY}\\
\hspace*{0.4 in} Astronomy \& Physics Honors, Core Distinction


\section{\sc \textcolor{redshade}{Publications} }
 \underline{\sl Refereed Publications} \\
{\bf \textcolor{blueshade}{Petersen, M. S.}}, Pe{\~n}arrubia, J. \href{https://ui.adsabs.harvard.edu/abs/2020MNRAS.494L..11P/abstract}{\it Reflex motion in the Milky Way stellar halo resulting from the Large Magellanic Cloud infall}, 2020, MNRASL, 494:11\\ % do we want citation counts?
{\bf \textcolor{blueshade}{Petersen, M. S.}}, Weinberg, M. D., and Katz, N. \href{https://ui.adsabs.harvard.edu/abs/2019MNRAS.490.3616P/abstract}{\it Using torque to understand barred galaxy models}, 2019, MNRAS, 490:3616\\
%{\bf Petersen, M. S.}, Gutermuth, R.A.,  Nagel, E.,  Wilson, G.W., Lane, J. \href{https://ui.adsabs.harvard.edu/abs/2019MNRAS.488.1462P/abstract}{\it Early science with the Large Millimetre Telescope: new mm-wave detections of circumstellar discs in IC 348 from LMT/AzTEC}, 2019, MNRAS, 488:1462\\
{\bf \textcolor{blueshade}{Petersen, M. S.}}, Katz, N. , \& Weinberg, M.D. \href{http://adsabs.harvard.edu/abs/2016PhRvD..94l3013P}{{\it The Dynamical Response of Dark Matter to Galaxy Evolution Affects Direct-Detection Experiments}}, Phys Rev D, 2016. Figure 4 was featured as part of the journal's `Kaleidescope'.\\
{\bf \textcolor{blueshade}{Petersen, M. S.}}, Weinberg, M. D., and Katz, N. \href{http://adsabs.harvard.edu/abs/2016MNRAS.463.1952P}{{\it  Dark matter trapping by stellar bars: the shadow bar}} 2016 MNRAS, 463:1952–1967.\\
%Bary, Jeffrey S. \& {\bf Petersen, M. S.} \href{http://adsabs.harvard.edu/abs/2014ApJ...792...64B}{{\it Anomalous Accretion Activity and the Spotted Nature of the DQ Tau Binary System}} 2014 MNRAS, 463:1952–1967.

%\vspace*{8pt}
%\underline{\sl Conference Proceedings} \\
%{\bf Petersen, M. S.}, Weinberg, M. D., and Katz, N. \href{http://adsabs.harvard.edu/abs/2014ASPC..480..157P}{{\it 
%Creation of Peanut-Shaped Bulges via the Slow Mode of Bar Growth}} ASPC, 480:157.

\underline{\sl Publications In Review} \\
{\bf \textcolor{blueshade}{Petersen, M. S.}}, Weinberg, M. D., and Katz, N. \href{https://ui.adsabs.harvard.edu/abs/2019arXiv190205081P/abstract}{\it Using commensurabilities and orbit structure to understand barred galaxy evolution}, arXiv e-prints\\
{\bf \textcolor{blueshade}{Petersen, M. S.}}, Weinberg, M. D., and Katz, N. \href{https://ui.adsabs.harvard.edu/abs/2019arXiv190308203P/abstract}{\it Using harmonic decomposition to understand barred galaxy evolution}, arXiv e-prints\\

%\underline{\sl Publications In Preparation} \\
%{\bf Petersen, M. S.}, Weinberg, M. D., and Katz, N. {\it Understanding trends in bar formation and evolution in varied $n$-body models} \\
%{\bf Petersen, M. S.}, Weinberg, M. D., and Katz, N. {\it Using the bar to drive radial mixing and bulge formation in a Milky Way-like galaxy} \\

%{\bf Petersen, M. S.}, Gutermuth, R.A., and Wilson, G.W. {\it Circumstellar Disk Masses in IC~348: Newly Detected Disks from LMT/AzTEC} \\
\pagebreak

%\section{\sc Selected Meetings \& Conferences}
%{\bf AAS 231, January 2018}, Washington, D.C.\\
%Dissertation: {\it A Modern Picture of Barred Galaxy Dynamics}

%{\bf APS April Meeting, January 2017}, Washington, D.C.\\
%Contributed Talk: {\it The Dynamical Response of Dark Matter to Galaxy Evolution Affects Direct-Detection Experiments}

%{\bf Great Lakes Cosmology Workshop, June 2016}, Hamilton, ON\\
%Contributed Talk: {\it Dark Matter Trapping by Stellar Bars: The Shadow Bar}

%{\bf What Shapes Galaxies?, April 2016}, STScI, Baltimore, MD\\
%Contributed Poster: {\it Dark Matter Trapping by Stellar Bars: The Shadow Bar}

%{\bf Structure and Dynamics of Disk Galaxies, August 2013}, Petit Jean Mountain, AR\\
%Contributed Talk: {\it Creation of Peanut-Shaped Bulges via the Slow Mode of Bar Growth}

%\section{\sc Selected Observational Experience}

%{\bf Large Millimeter Telescope}, PI (Early Science 2,3,4 2014-2016), 60 hours\\
%{\it Circumstellar Disk Masses in IC~348}\\
%Using AzTEC, a 1.1 mm camera.

%{\bf KPNO 0.9m}, PI (2016-2017), 5 nights; Co-I (2014-2018), 30 nights\\
%{\it Deep Imaging of Nearby Low Surface Brightness Disks}\\
%{\it Ionization States of Green Pea Galaxies}\\
%Using the Half-Degree Imager (HDI).
% would be nice to put the Kim title in here as well

%{\bf Gemini South}, Co-I (2017A), 18 hours\\
%{\it High-Resolution Spectroscopy of Orbitally-Modulated Accretion Activity in Pre-Main Sequence Binaries.} Using PHOENIX.

%{\bf APO 3.5 m}, Co-I (2017B), 4 nights\\
%{\it Optical to near-infrared Monitoring of Eccentric Pre-Main
%  Sequence Binaries.} Using ECHELLE and TripleSpec.

%{\bf NASA IRTF}, Co-I (2018B), 5 nights\\
% {title?}


\section{\sc \textcolor{redshade}{Advising}}

% add MPhys once it stars

% add titles of projects? and progress?
{\bf Senior Honours Research Advisor} \hfill\textcolor{grey}{2019-}\\
Designed and advised three research projects for undergraduate students at the University of Edinburgh.\\

{\bf Research Advisor} \hfill\textcolor{grey}{2020}\\
Designed, sought funding for, and advised two research projects for undergraduate students at the University of Edinburgh.


%{\bf Research Mentor, 2016-2017}\\
%Mentored and advised Amelia Vinciguerra (UMass 2017) in her senior undergraduate research project in collaboration with Martin Weinberg.

\section{\sc \textcolor{redshade}{Service}}

{\bf ROE Seminar Organiser} \hfill\textcolor{grey}{2019-}\\
Responsible for selection of speakers and organising delivery of talks for the Royal Observatory. Includes remote organisation and hosting during work-from-home period.\\

{\bf Local Universe Reading Group Organiser} \hfill\textcolor{grey}{2019-}\\
Responsible for programming and hosting a roughly dozen-person reading group coverinng multiple research teams at the ROE.

\section{\sc \textcolor{redshade}{Teaching}}

{\bf Five College Astronomy Teaching Assistant} \hfill \textcolor{grey}{2014-2018}\\
Assisted Professor James Lowenthal, Dr. Anne Jaskot, and Dr. Kim Ward-Duong in Observational Techniques I/II, a two-semester observing course involving a yearly observational component at Kitt Peak National Observatory. University of Massachusetts Distinguished Teaching Award finalist.\\

{\bf Continuing Education Head Instructor} \hfill \textcolor{grey}{2013-2018}\\
Developed a two-week intensive continuing education class for the University of Massachusetts, then reshaped the class into a program involving many graduate students.

%
%{\bf Public Outreach Lecturer, 2013-present}\\
%Advised the undergraduate Astronomy club. Lead semesterly star walks
%for dormitories. Offered yearly open night sky observing sessions.

%{\bf Python Instructor, 2014-present}\\
%Provided Python instruction and support to the undergraduate Five College summer undergraduate researchers.

\section{\sc \textcolor{redshade}{Professional Links}}
{\bf Research Webpage} \url{https://michael-petersen.github.io}

{\bf Github Code Repository} \url{https://github.com/michael-petersen}


\section{\sc \textcolor{redshade}{References}}

{\bf Jorge Pe{\~n}arrubia}\\
Postdoctoral Research Associate advisor.

{\bf Martin D. Weinberg}\\
Primary dissertation advisor.

{\bf Neal Katz}\\
Co-dissertation advisor.


\end{resume}
\end{document}